% プロジェクト学習中間報告書書式テンプレート ver.1.0 (iso-2022-jp)

% 両面印刷する場合は `openany' を削除する
\documentclass[openany,11pt,papersize]{jsbook}
  
  % 報告書提出用スタイルファイル
  %\usepackage[final]{funpro}%最終報告書
  \usepackage[middle]{funpro}%中間報告書
  
  % 画像ファイル (EPS, EPDF, PNG) を読み込むために
  \usepackage[dvipdfmx]{graphicx,color}
  
  % ファイル分割のためのパッケージ
  \usepackage{subfiles}
  
  % ここから -->
  \usepackage{calc,ifthen}
  \newcounter{hoge}
  \newcommand{\fake}[1]{\whiledo{\thehoge<70}{#1\stepcounter{hoge}}%
    \setcounter{hoge}{0}}
  % <-- ここまで 削除してもよい
  
  % 年度の指定
  \thisYear{2017}
  
  % プロジェクト名
  \jProjectName{ビーコンIoTで函館のまちをハックする}
  
  % [簡易版のプロジェクト名]{正式なプロジェクト名}
  % 欧文のプロジェクト名が極端に長い(2行を超える)場合は、短い記述を
  % 任意引数として渡す。
  %\eProjectName[Making Delicious curry]{How to make delicious curry of Hakodate}
  \eProjectName{Leverage the Beacon IoT in Hakodate Real Downtown for Our Smarter Life}
  
  
  % <プロジェクト番号>-<グループ名>
  \ProjectNumber{8-A}
  
  % グループ名
  \jGroupName{Hako-B}
  \eGroupName{Hako-B}
  
  % プロジェクトリーダ
  \ProjectLeader{1015253}{橋場保鷹}{Hodaka~Hashiba}
  
  % グループリーダ
  \GroupLeader  {1015053}{佐藤秀輔}{Shusuke~Sato}
  
  % メンバー数
  \SumOfMembers{4}
  % グループメンバ
  \GroupMember  {1}{1015157}{小笠原瑠奈}{Runa~Ogasawara}
  \GroupMember  {2}{1015050}{北原康太}{Kota~Kitahara}
  \GroupMember  {3}{1015204}{小島雄士}{Yuji~Kojima}
  \GroupMember  {4}{1015053}{佐藤秀輔}{Shusuke~Sato}
  
  % 指導教員
  \jadvisor{松原克弥,藤野雄一,鈴木恵二,奥野拓}
  % 複数人数いる場合はカンマ(,)で区切る。カンマの前後に空白は入れない。
  \eadvisor{Katsuya~Matsubara,Yuichi~Fujino,Keiji~Suzuki,Taku~Okuno}
  
  % 論文提出日
  \jdate{2017年7月26日}
  \edate{July~26, 2017}
  
  \begin{document}
  %
  % 表紙
  \maketitle
  
  %前付け
  \frontmatter
  
  % 和文概要
  \begin{jabstract}
  
  % プロジェクト全体の日本語概要
  \subfile{common/common-jabstract}
  
  % プロジェクト全体の日本語概要
  
  % 和文キーワード

  
  %英語の概要
  \begin{eabstract}
  
  % プロジェクト全体の英語概要
  \subfile{common/common-eabstract}
  
  % グループの英語概要
  
  % 英文キーワード

  
  \tableofcontents% 目次
  
  
  \mainmatter% 本文のはじまり
  
  % プロジェクト共通の項目
  \subfile{common/common-chapters}
  
  
  
  % グループごとの項目
  
  \chapter{本グループについて}


  % 以降、付録(付属資料)であることを示す
  \begin{appendix}
  
  \chapter{中間報告会で使用した本グループのポスター}
  
  \begin{figure}[thbp]
    \begin{center}
      \includegraphics[width=0.92\hsize]{img/poster_v44.pdf}
      \caption{サプライズスナップポスター(中間報告会)}
      \label{fig:sup_mid_poster}
    \end{center}
  \end{figure}
  
  %付録の終わり
  \end{appendix}
  
  
  %\backmatter
  
  \begin{thebibliography}{9}
    \bibitem{JohoWhitepaper} 総務省, 平成28年版 情報通信白書 第1部 特集 IoT・ビッグデータ・AI~ネットワークとデータが創造する新たな価値~
    http://www.soumu.go.jp/johotsusintokei/whitepaper/ja/h28/html/nc122530.html
  
    [last accessed 2017/7/24]
  \end{thebibliography}
  
  \end{document}
  